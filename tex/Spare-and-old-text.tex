







Some argue that market risk is small for public sector projects:
\begin{quote}
\textquotedblleft {\small There remains the pure risk not diversified away because it arises from correlation between project benefits and general economic activity... For the typical public-sector project, the required adjustment to benefits turns out to be very small in relation to the margin for error in CBAs.''}%
	\footcite[][v]{BITRE2005RiskinCostBenefitAnalysis}
\end{quote}

This perspective is contrasted with the view that, unless there is evidence to the contrary, it is best to assume that public sector projects are no less risky, because:
\begin{quote}
\textquotedblleft {\small The consensus view is that most government projects are highly correlated with return to the economy as a whole.''}%
	\footcite[][60]{ Harrison-Valuing-the-Future}
\end{quote}


\begin{bigbox}{Journal of Cost-Benefit Analysis discount rate debate about the displacement impacts of public investments}{box:journal-of-CBA-debate-about-discounting}
In 2013, the Journal of Cost-Benefit Analysis published a series of papers covering the debate between advocates of differing discounting approaches. The debate hinged largely on one issue – whether public investments displaced investment or consumption. 

[\hl{We might not need this box?}]

The side of the debate that favoured lower discount rates, argued that:

\begin{quote}
\textquotedblleft {\small A more realistic assumption is that increased government spending is funded by increased taxes. While some projects at the margin might be financed by debt, all projects are ultimately funded by taxes. Furthermore, taxes primarily reduce current consumption rather than private investment for the simple reason that consumption is much larger than investment (typically five times as large) and, therefore, taxes on investment cannot yield as much as taxes on consumption.”}
\end{quote}
The other side argued that higher discount rates were appropriate, on grounds that: 

\begin{quote}
\textquotedblleft {\small The appropriate measure … is the social opportunity cost of borrowed funds, not the social opportunity cost of funds raised by an increase in the income tax or some other broad based tax. Admittedly, it is empirically challenging to arrive at reliable estimates of the weights … (i.e., the proportions of an increment of borrowed funds that displace investment versus consumption and, in an open economy, net exports), but the consensus of those who have looked carefully at the matter is that investment is much more sensitive to the rate of return than consumption or net exports, so the bulk of an increment in borrowed funds displaces investment.”}
\end{quote}

The challenge in arriving at an empirically-justified solution was acknowledged by both sides.

\begin{quote}
\textquotedblleft {\small Because both … approaches are normative and do not generate testable hypotheses that would allow for empirical falsification, the choice of the correct method for choosing a [discount rate] largely depends on what opportunities one believes are sacrificed when a public investment project goes forward: lower taxes and more current private consumption [the ``low rate'' advocates' view], or lower government debt, lower interest rates and more current private investment [the ``high rate'' advocates' view].”}
\end{quote}

This debate serves as an illustration of the entrenched views among experts, which are unresolved empirically.

\boxsources{\textcite{Moore-Boardman-Vining-2013a-More-appropriate-discounting}, \textcite{Burgess-Zerbe-2013-The-most-appropriate-discount-rate} and \textcite{Moore-Boardman-Vining-2013b-The-choice-of-discount-rate}}
\end{bigbox}


A standard discount rate of 7 per cent has been entrenched in almost all Australian jurisdictions for some time, due to what appears to be a combination of pragmatism and inertia.

The 7 per cent discount rate seeks to quantify the returns from the next best use of the resources, and to do so by inferring these returns from private sector market returns. In other words, a 7 per cent discount rate indicates that the government could, instead of building a new road or rail line extension, have invested the money in the market and received a return of 7 per cent.

But even within the framework of assessing the value of a public infrastructure investment against private sector returns, strong assumptions are needed to reach a rate of 7 per cent – particularly a rate of 7 per cent that does not vary over time or between projects.

This chapter examines three key claims that must be believed for the rate of 7 per cent to be defensible, and finds significant weaknesses with each. The first claim is that a public infrastructure investment leads private sector investors to invest less than they otherwise would. The second claim is that even if governments do not invest in higher-yielding projects, what matters is that they could. And the third claim is that public sector infrastructure projects offer strong returns in good economic times and weak returns in bad times, just like the returns to the market as a whole. The following sections explain the weakness of each of these claims. 

\section{Claim 1: that public infrastructure investment leads private investors to invest less}

Many commentators take the view that public sector projects entirely, or almost entirely, displace private sector investment. The mechanism by which they do this is employing workers and equipment, using input materials and finance, so that there is less available for the private sector to use on the next best project. Proponents of this view argue that “market rates reflect the opportunity cost of investing in public projects”.%
    \footcite[][viii]{Harrison-Valuing-the-Future}  

But this assertion is not backed by evidence. In fact, recent Australian history shows the opposite situation - simultaneous booms in private and public sector investment. The following two sections explain these shortcomings.

\subsection{No evidence for the claim that public investment reduces private investment}

Australian commentators appear to have accepted the notion that public investment reduces private investment, even though there is no evidence for this idea. \textcite{Harrison-Valuing-the-Future} cites a range of sources to support the notion that public investment predominantly displaces private investment.%
    \footcites[][32]{Harrison-Valuing-the-Future}[][xx - check - for alternative perspective]{Boardman-Cost-Benefit-Analysis-4th-Edition}
However, none of the sources provide factual evidence, instead offering general assertions.

One of the key sources behind the claim is a 1972 book by Arnold Harberger, in which he draws on his judgment of “the accumulated econometric evidence on investment functions” to reach the conclusion that there is “a reasonable presumption that the relevant weighted average will be reasonably close, if not precisely equal, to the marginal productivity of capital in the private sector”.%
    \footcite[][108]{Harberger-Project-Evaluation-1972}  

Another source cited in support of investment-based discount rates is the Department of Finance’s Handbook of Cost-Benefit Analysis, which states “the general conclusion (and a common international practice) is that a producer rate of discount is the appropriate rate of discount to employ”.\footcite[][XX]{Dept-Finance-2006-Handbook-of-CBA}  Similar to this, a 2003 textbook states “the usual and safe presumption is that all the capital employed represents investment foregone”.\footcite[][XX]{Abelson-2003-public-economics}  

The type of primary empirical evidence required to help answer the question is hard to come by. However, it is clear that the case that private sector investment is displaced is far from robust. 

And notwithstanding these authorities considering the matter settled, there is plenty of debate and disagreement internationally over whether public investment can be thought to displace private investment. \Vref{box:journal-of-CBA-debate-about-discounting} sets out a debate over these issues published in the Journal of Cost-Benefit Analysis in 2013 that illustrates the contested nature of the debate.

Given the absence of empirical evidence in the literature, the following section turns to recent Australian experience for evidence that an extra dollar of public sector investment reduces private sector investment by a dollar or close to it. 

\subsection{Recent Australian history shows simultaneous public and private investment booms}

The past 12 years has seen high levels of investment in transport infrastructure. These are the highest since records were first collected by the ABS in 1987, and peaked in 2010. 

But there is no evidence that private sector investment diminished in this period. On the contrary - at the same time as record public investment, there was also exceptionally high private sector investment associated with the mining investment boom. Mining investment reached 8 per cent of GDP at its peak, where it would otherwise have been closer to 3 per cent.

[insert new chart of RBA chart 2 page 18 overlaid with eng work done for public sector]

A small open economy like Australia handles a period of high infrastructure investment by drawing in workers, materials and equipment that would otherwise be doing something else - or be unemployed - in Australia, or bringing them in from overseas. If Australian governments finance the import of workers, materials and equipment by borrowing, then the debt repayments will have impacts on future government spending, or the amount of tax levied, or both.

Australia not only had simultaneous public infrastructure investment and mining investment booms, but also lifted consumption at the same time.\footcite[][20]{RBA-2014-effect-of-mining-boom}  

In summary, recent experience shows no impediment to exceptionally high levels of public investment coinciding with exceptionally high levels of private investment, plus a substantial step up in consumption. 

\section{Claim 2: That what government could have done matters more than what it would have done}

Using a market rate of return to quantify the opportunity cost of a public investment is sometimes expressed as an argument that there is no case for allocating resources to low-return investments when higher returns are available.\footcite[][viii]{Harrison-Valuing-the-Future} 

But this approach focuses on what governments \emph{could} do, unmoored from the reality of what they \emph{actually} do, or why they do it. In fact, a society and economy like Australia relies on a combination of public and private infrastructure, both because public investment makes private investment possible, and also because public investment can improve the returns on private investment.

Private sector activity is wholly reliant on functioning roads, railways, telecommunications, water and power networks, as well as the intangible infrastructure of the legal system, defence and education systems. Governments often build new infrastructure in greenfields sites, enabling property developers and businesses to start operating in those locations. 
Transport infrastructure investments exhibit this character very strongly. They unlock new places, and they enable redevelopment and repurposing of existing places, particularly in major cities. Further, a “transport improvement will generally change the pattern of private investment across locations, and this process of encouraging – or even ‘unlocking’ – private development is often put forward as one of the major impacts of transport projects”.\footnote{venables reference} 

\section{Claim 3: The returns on this project will be like those on the average private sector project}

The most important decisions for an investor working out the value of an asset is whether the one they choose will yield returns in good times and bad. While investments on average will track the market as a whole, some will track it closely, while others will yield returns that are impervious to fluctuations in the market or the economy.\footnote{In finance terms, the extent of this tracking of general market returns is known as the project’s ‘beta’.  Beta reflects the impact on the investment of economy-wide risks, or ‘systemic risks’, such as the risk of a recession. It does not reflect the impact of project-specific risks, such as poor management or cost overruns, because these risks are generally managed by holding a portfolio of investments where investments with higher returns balance those with lower returns, on average.}
Projects that yield returns regardless of how strong the economy is are desirable investments because the investor can rely on them yielding steady returns, and because those returns are more valuable to the investor in weak economic times, when investment returns are harder to get. 

The view that appears to have prevailed in Australian discounting practice is that returns on government infrastructure projects track returns on investments across the market as a whole:

\begin{quote}
 “the consensus view is that most government projects are highly correlated with returns to the economy as a whole”.\footcite[][60]{Harrison-Valuing-the-Future} 
\end{quote}

But this view does not appear to be supported, either by logic or empirically. The rest of this section gives two reasons why not.

\subsection{Public infrastructure projects are different to the average investment}

Governments have traditionally provided much of the community’s infrastructure, in the form of roads, water distribution networks, energy generation and distribution networks, ports, airports and so on. Public infrastructure is generally long-lived, often a monopoly, commonly provides services that the community sees as essential, and may offer benefits that do not readily translate into financial returns.

None of these characteristics definitively separates public from private infrastructure. Indeed, governments have clearly reduced their direct ownership of previously public infrastructure assets, through privatisations and long-term leases.

Nevertheless, there remain observable differences between public infrastructure assets and other investments. An analysis of Infrastructure Australia’s infrastructure priority list shows x is urban roads, y is urban rail and z is … By contrast, an analysis of the Deloitte Investment Monitor shows that, of the projects built by all sectors over the past 15 years, x are urban roads, y are urban rail and z are xxx. 

\subsection{Government regulators regularly find government infrastructure to be relatively invariant to general market conditions}

In several sectors of the economy, regulators determine the extent to which the returns on government assets track those of the market as a whole. For instance, the Australian Energy Regulator performs this function for electricity transmission and distribution networks as part of establishing revenue caps. Similarly, the NSW Independent Pricing and Regulatory Tribunal performs this function to establish a rate of return for rail and water assets.

These regulators find that government assets do not have a close relationship to general market returns. Not surprisingly, the basic services that government infrastructure tends to provide are used with only modest variation according to the state of the economy as a whole. People still use water, power and rail services even when the economy is weak.\footnote{IPART estimates beta for the rail infrastructure sector to be between 0.7 and 1.}  Even in the unfortunate event of a recession, a doubling in unemployment would have minimal impact on most people’s commuting patterns, nor their education, leisure and other activities – which determines the demand for transport infrastructure.

This is at odds with the view that most government projects have returns that are highly correlated with the returns on the market as a whole. As far as regulated assets are concerned, this is not the case.

The advice to potential investors by investment advisers supports the view that infrastructure assets, particularly those that have been in the public sector, are investments that have limited relationship with general market conditions.\footnote{add references; chase buy side advisors}    

A more defensible view is that market risk is very small for public sector projects:

\begin{quote}
“There remains the pure risk not diversified away because it arises from correlation between project benefits and general economic activity… For the typical public-sector project, the required adjustment to benefits turns out to be very small in relation to the margin for error in CBAs.”\footcite[][v]{BITRE2005RiskinCostBenefitAnalysis} 
\end{quote}





\begin{bigbox}{Journal of Cost-Benefit Analysis discount rate debate about the displacement impacts of public investments}{box:journal-of-CBA-debate-about-discounting}
In 2013, the Journal of Cost-Benefit Analysis published a series of papers covering the debate between advocates of differing discounting approaches. Just as was suggested at the start of this chapter, the debate hinged largely on one issue – whether public investments displaced investment or consumption. 

The side of the debate that favoured lower discount rates (based on the social time preference approach), argued that:

\begin{quote}
\textquotedblleft A more realistic assumption is that increased government spending is funded by increased taxes. While some projects at the margin might be financed by debt, all projects are ultimately funded by taxes. Furthermore, taxes primarily reduce current consumption rather than private investment for the simple reason that consumption is much larger than investment (typically five times as large) and, therefore, taxes on investment cannot yield as much as taxes on consumption.”
\end{quote}
The other side argued that higher (opportunity cost of capital) discount rates were appropriate, on grounds that: 

\begin{quote}
\textquotedblleft The appropriate measure … is the social opportunity cost of borrowed funds, not the social opportunity cost of funds raised by an increase in the income tax or some other broad based tax. Admittedly, it is empirically challenging to arrive at reliable estimates of the weights … (i.e., the proportions of an increment of borrowed funds that displace investment versus consumption and, in an open economy, net exports), but the consensus of those who have looked carefully at the matter is that investment is much more sensitive to the rate of return than consumption or net exports, so the bulk of an increment in borrowed funds displaces investment.”
\end{quote}
The challenge in arriving at an empirically-justified solution was clearly well-acknowledged by both sides.

\begin{quote}
\textquotedblleft Because both … approaches are normative and do not generate testable hypotheses that would allow for empirical falsification, the choice of the correct method for choosing a SDR largely depends on what opportunities one believes are sacrificed when a public investment project goes forward: lower taxes and more current private consumption \hl{(the STP-SPC view)}, or lower government debt, lower interest rates and more current private investment (the SOC view).” 
\end{quote}

This debate serves as an illustration of the entrenched views among experts, which have failed to be resolved through empirical analysis.

\boxsources{\textcite{Moore-Boardman-Vining-2013a-More-appropriate-discounting}, \textcite{Burgess-Zerbe-2013-The-most-appropriate-discount-rate} and \textcite{Moore-Boardman-Vining-2013b-The-choice-of-discount-rate}}
\end{bigbox}






\chapter{\hl{OLD VERSION} What are the primary considerations in setting a discount rate for transport projects?}

\section{Developing a rule to achieve multiple objectives}
As \Vref{box:<what-is-the-discount-rate>} explained, society places a higher value on benefits and costs that occur in the near future, compared with those that occur at a later date.

Discounting requires policymakers to develop a rule that coherently adjusts costs and benefits that occur in the future.%
    \footnote{Clearly, a range of different perspectives can inform such a rule, because valuing the impacts of policies today on future generations involves questions of ethics and morality. The explicit framing we adopt here is that of economic cost-benefit analysis. In adopting this lens, which we acknowledge to be a value judgement (\textcite{Creedy-2007-Polic-Evaluation-A-Reminder}), we note that the results of any cost-benefit analysis do not provide policymakers with a ``universally correct'' assessment of a project's merits, but rather with an answer from the perspective of economic efficiency.}
This is the process of assessing the economy-wide impacts of government investment or policy change through economic cost-benefit analysis. 

Cost-benefit analysis assesses the stream of costs and benefits that accrue to people in a society in terms of what they consume.%
    \footcite[][xii]{Harrison-Valuing-the-Future}
The discounting rule should therefore explain how to value changes in future generations' consumption that occur as a result of different projects or regulations under consideration today. If a transport project makes us richer, thereby giving us greater future consumption than we would otherwise have have, by how much should that future increase in consumption be adjusted to make it consistent with how we value consuming now versus later in other settings?  

In addition to achieving this objective, the discounting rule also needs to account for any higher returns that are foregone if private investment is displaced by the public sector project. This is because these higher returns would have made it possible to have had a larger volume of consumption in future. The transport project being assessed should leave society with at least the same amount of overall consumption -- and ideally more.%
    \footcite[][10]{Harberger-and-Jenkins-2015-musings-on-the-discount-rate} 

These two objectives of the discounting rule -- helping us understand how the changes in future generations' overall consumption should be valued today, together with acknowledging the higher consumption that private investment might have made possible -- are the source of much debate about the appropriate way to set a discount rate.%
    \footcite[][46]{Parker-2009-NZTA-discount-rates}
This chapter focuses on the the implications of these two objectives for how we set the discount rate. 


\section{A view about what is ``given up'' when governments invest in transport projects determines the choice of discount rate}

What society ``gives up'' when public projects are undertaken provides an insight into how future impacts might be valued -- i.e. what the appropriate rate of discount should be. 

If private sector investments are foregone, we should compare the impacts of transport projects with the (larger) amount of consumption that these investments would have created. If private consumption is foregone when government undertakes a transport project, then we should compare the project's returns with how people trade-off present and future consumption in other settings. 

Alternatively, neither investment or consumption may be foregone in the present, if foreign borrowing is used to fund the transport project. In this case, the impact on reduced consumption or investment is deferred to a future period when the foreign borrowings are repaid.

In the sections below we look at the scenarios that would lead to upper and lower bounds on the discount rate -- where they \emph{entirely} displace investment and consumption, respectively. 


\subsubsection{If private sector investment is reduced when a transport project is undertaken, a discount rate around 8 per cent is appropriate}

If a public infrastructure project led to the private sector investing less than it otherwise would, the discount rate should be inferred from an estimate of the average private sector return on capital. 

The literature identifies many ways to estimate the appropriate discount rate to use when a public sector project comes at the expense of private sector investment. This includes using the capital asset pricing model,%
    \footnote{New Zealand Treasury}
the weighted average cost of capital,%
    \footnote{IPART (unpublished)}
returns to investments in the share market,%
    \footnote{Boardman et al}
and national accounts measures of the return to private capital.%
    \footnote{Harrison}

The literature on the merits of each measure of private sector investment returns is both exhaustive and varied. Each measure has strengths and weaknesses. In addition to a lack of consensus about the appropriate measure, other complications include: 
\begin{itemize}
    \item the historical period over which the benchmark is calculated (whether a 10, 20, 30 year, or longer time period is used); 
    \item the extent to which the benchmark should be forward-looking rather than entirely based historical on historical data; and 
    \item how it should be adjusted for the risk that is inherent in measures of average private sector investment returns (these issues, which are also the subject of great debate and confusion, are set out in more detail in \Cref{box:discount-rate-and-risk}).
\end{itemize}
    
In practice, the consensus of estimates has led to discount rates in the range of 6-8 per cent (after adjusting for inflation).%
    \footcite[][5]{Argyrous-ANZSOG-review-of-CBA-guidelines}
The current use of 7 per cent as the central discount rate relies on this assumption.% 
    \footnote{For example, NSW Treasury's guidance states that ``The theoretical basis for the long term social discount rate used in this Guide is the opportunity cost of capital. This recognises that any Government initiative can occur at the expense of other alternative public investment or private investment.'' (\textcite{New-South-Wales-Treasury-2017-CBA-Guidelines}).}
Recent Grattan analysis of national accounts estimates of private sector returns to capital confirms this, finding that long term returns to capital in the order of 8-9 per cent.% 
    \footnote{Appendix A provides further analysis of the historical returns to capital using Australian System of National Accounts.}


\begin{bigbox}{How should the discount rate and incorporate risk?}{box:discount-rate-and-risk}

Under the assumption that private investment is reduced when a public project is undertaken, the choice of an appropriate discount rate  is ultimately a technical decision relating to risk.%
	\footcite[][27]{Grimes-Beyond-Simple-CBA}
But there are two types of risk that need to be distinguished. 

The first type is project-specific risk. This is the risk that the estimated benefits and costs of a project do not materialise, or differ from their forecasts. We know from our own research into cost overruns in transport projects that this risk is not well-handled in project evaluations.%
	\footcite{Terrill-etal-2016-Cost-overruns-in-transport-infrastructure}

But this type of risk is best handled through assigning probabilities to different possible outcomes and estimating expected values, rather than through manipulations of the discount rate.%
	\footcite[][v]{BITRE2005RiskinCostBenefitAnalysis}

The other type of risk is market-related risk. Assets whose returns fluctuate positively with consumption make consumption more volatile and investors need a higher expected return to be induced to hold them. That is, their rate of return includes a market risk premium to compensate for the cost of the risk they impose on investors, because this risk to an investment's expected benefits cannot be diversified. 

But here, however, experts often disagree. 

Some argue that market risk may be small for public sector projects:
\begin{quote}
\textquotedblleft {\small There remains the pure risk not diversified away because it arises from correlation between project benefits and general economic activity... For the typical public-sector project, the required adjustment to benefits turns out to be very small in relation to the margin for error in CBAs.''}%
	\footcite[][v]{BITRE2005RiskinCostBenefitAnalysis}
\end{quote}

This is contrasted with the view that, unless there is evidence to the contrary, it is best to assume that public sector projects are no less risky, because:
\begin{quote}
\textquotedblleft {\small The consensus view is that most government projects are highly correlated with return to the economy as a whole.''}%
	\footcite[][60]{ Harrison-Valuing-the-Future}
\end{quote}

In addition, separate from these disagreements, other unresolved issues remain. For example, what is the appropriate risk premium to include in the discount rate when a project's up-front costs are risky but its longer-term benefits are certain (or, more generally, follow a different risk profile)? In that case, should the costs be discounted assuming an average private sector risk premium, while the benefits should be discounted at a risk-free rate?%
	\footcite[][26]{Grimes-Beyond-Simple-CBA}

In our view, the inclusion of a risk premium into cost of capital based discount rates seems appropriate, because it is important to avoid situations where a public sector can justify a wider range of projects merely on the assumption that its cost of risk is lower than the private sector's. But these disagreements highlight to us that the issues are far from clearly resolved with academia. \Cref{chap:what-is-wrong-with-seven-per-cent} looks more closely at what market risk premium might be appropriate to assume in the context of transport project evaluations. 

\end{bigbox}


\subsubsection{Projects that displace private sector consumption should use a discount rate around 3 per cent}

Instead of investing less, people may simply consume less when public sector projects are undertaken. If this is the case, then the benefits that a transport project delivers -- which, as noted above, are always measured as changes in future consumption -- should not be compared with the rate of return on private sector investment. 

The rate of discount that consumers face is determined by the way people value consuming today compared with consumption in the future. This rate differs from an investment rate because taxes and market distortions drive a gap between what investments earn and what consumers receive.%
    \footcite[][26]{Harrison-Valuing-the-Future}

In practical terms, there are two options for identifying the ``consumption'' rate of discount. The first looks at the best return that many individuals are able to earn in exchange for postponing consumption, which in many cases is the real after-tax return on savings.%
    \footcites[][239]{Boardman-Cost-Benefit-Analysis-4th-Edition}[][xii]{Harrison-Valuing-the-Future}

The second option eschews the use of private market interest and investment return rates, and instead develops a parameter called the ``rate of social time preference'' as the discount rate. This approach has gained prominence in recent years through its application in analysis of policies to address climate change (see \Cref{box:social-time-preference-approach} for its rationale and composition). While this approach has a range of advantages -- particularly in policy settings where reference to private market returns is dubious, such as policies that have impacts extending into the distant future -- its analytical complexity is the main reason that we do not propose using it in the context of transport projects. 

As with finding a precise value for the discount rate when a public project results in less private investment, it is also not easy to identify one universally correct value of the consumption rate of discount. 

In practice, the economic literature identifies consumption discount rates in the range 2-4 per cent.%
    \footnote{\textcite{Boardman-Cost-Benefit-Analysis-4th-Edition}, \textcite{Harrison-Valuing-the-Future}, \textcite{Parker-2009-NZTA-discount-rates}}. 
Of course, at present the returns to savings are much lower -- before taxes and inflation are factored in, term deposits currently return between 2-3 per cent, meaning that the after-tax and after-inflation returns are only marginally greater than zero at present. It remains important, however, to look to the longer term historical return to savings rather than only looking current rates. As such here we view the most appropriate consumer rate of discount to be around 2 per cent.


\begin{bigbox}{The discount rate in climate change analysis -- using the social time preference rate}{box:social-time-preference-approach}

Readers familiar with the discounting literature will have encountered a discounting approach based on the rate of ``social time preference'' -- in other words, the preference of a society for consuming today versus consuming later. 

This approach estimates an equation with two basic components. The first component discounts future levels of consumption because of our impatience -- noting that we give less weight to those alive in the future than we give to ourselves. The second component is based on the expectation that future generations will be richer (and therefore have greater levels of consumption), and that the additional increment of wellbeing they receive from consumption diminishes as their wealth grows over time.%
    \footcite{Summers-and-Zeckhauser-Policymaking-for-posterity}

This approach provides an alternative way of assessing the rate of discount that consumers use when making decisions about consuming today or consuming in the future. 

One of the main motivations for its use is a distrust of discount rates that are inferred from private markets. This distrust is valid -- it is, for example, difficult to identify a single ``consumption'' rate of interest from the wide-range of interest rates that consumers face. People's preferences vary widely, and often exhibit internal inconsistencies -- such as where people have outstanding credit card debt at the same time as they have other savings accounts and pay off a mortgage.%
    \footcite[][29]{Harrison-Valuing-the-Future}

In addition, the logic of using discount rates inferred from private market returns becomes far less compelling when the impacts arise centuries into the future -- as is the case for climate change policies. In such cases, ``the benefit-cost rationale for discounting breaks down and must be reestablished on principles incorporating intergenerational equity.''%   
    \footcite[][S-8]{Lind-1990-Reassessing-the-governments-discount-rate}

Importantly, under this approach, where there is reason to believe that private sector investment will be displaced by a public sector project, this needs to be incorporated into the analysis separately from the discount rate. It is done on a case-by-case basis by applying an additional weighting, known the `shadow price of investment', to any estimates of displaced private sector investment.% 
    \footnote{Boardman et al,  page 239.}

While we support using approach in certain contexts, such as climate change policies, we don't recommend using it to set the discount rate for transport project assessment for several reasons. First, the analytical complications of appropriately assessing whether private sector investment will be displaced by a public sector project are formidable. Second, most of the impacts of transport projects are felt within time-frames -- between 30 and 100 years -- that allow us to form reasonable assumptions about likely private investment and consumption rates of discount using historical private market returns and interest rates. Third, the ethical dimensions that the choice of an appropriate ``pure time preference rate'' raises -- i.e. exactly how selfish society is or should be -- are in many ways beyond the scope of cost-benefit analysis based on economic efficiency. 

Instead, sensitivity testing of a transport project's economic merits using a social time preference approach seems a reasonable way to give decision-makers this perspective as additional information rather than allowing it to become the core result.

\end{bigbox}


\subsubsection{Projects that displace \emph{both} private investment and consumption should use a rate that acknowledges both}

The two scenarios outlined above -- where the effects of transport projects are \emph{entirely} on reducing private sector investment \emph{or} on reducing consumption -- are clearly upper and lower bounds. As such, much of the discounting literature proposes using the proportions of both investment and consumption that are ``given up.''%
    \footnote{``Thus, we can conclude that the public sector's discount rate should be a weighted average of the rate facing consumers and the tax-distorted rate used by firms, the weights being the compensated interest derivative of consumption and the interest derivative of private investment, respectively'' (\textcite[][401]{Sandmo-and-Dreze-1971-Discount-Rates}).}
It can also be broadened to incorporate the extent to which the public sector project will be financed by foreign capital, given that in an open economy capital is internationally mobile, meaning that increased demand for capital can be met from increased capital funds from abroad.

This approach has intuitive appeal, but big questions -- and disagreements -- remain about whether it is possible to make reasonable assumptions about these proportions. 

The next chapter explores this question of the appropriate weights for displaced investment and displaced consumption in more detail, to assess whether the prevailing use of 7 per cent as the central discount rate is appropriate.



















\chapter{OLD VERSION What's wrong with a 7 per cent discount rate?}\label{chap:what-is-wrong-with-seven-per-cent}

Australian practice has settled in recent years on a discount rate of 7 per cent, based on a cost of capital approach.%
    \footnote{As set out in \Vref{section:australian-practice-lacks-coherence}.}

This chapter describes in \Cref{section:Public-investment-may-not-displace-private-investment} that the evidence base supporting this approach is very weak -- in short, public investment may not displace private investment. This should not be surprising -- evidence for the displacement effects of public investments is is hard to come by. \Cref{section:examples-of-displacement-effects-of-transport-projects} then sets out a range of examples to demonstrate the difficulties facing policymakers in discerning the displacement effects of most transport projects.

In combination, \Cref{section:Public-investment-may-not-displace-private-investment} and \Cref{section:examples-of-displacement-effects-of-transport-projects} reveal the need for pragmatic solutions to discount rate setting. 

In \Cref{section:downside-pressures-on-a-weighted-discount-rate}, we extend the discussion to make a range of claims about why the discount rate might lie to the lower end of the plausible ranges typically used in practice. Unlike the claims made in \Cref{section:Public-investment-may-not-displace-private-investment}, we acknowledge that these arguments remain subject to debate. 


\section{Claims that public investments displace private investments have no basis in fact}\label{section:Public-investment-may-not-displace-private-investment}

The use of relatively high opportunity cost of capital discount rates is based on the claim that public infrastructure investment predominantly displaces private sector investment. This section examines the evidence for this claim. 

\textcite{Harrison-Valuing-the-Future}, which provides one of the most thorough and clear arguments in favour of adopting a relatively high opportunity cost of capital discount rate, cites a range of sources to support the notion that public investment predominantly displaces private investment. However, none of the sources provide factual evidence, instead offering general assertions.

One of the key sources behind the claim is a 1972 book by Arnold Harberger, in which he draws on his judgement of “the accumulated econometric evidence on investment functions” to reach the conclusion that there is “a reasonable presumption that the relevant weighted average will be reasonably close, if not precisely equal, to the marginal productivity of capital in the private sector.”  This view, formed over 45 years ago, should not be considered a sufficient evidence base to settle the question. 

Another source cited in \textcite{Harrison-Valuing-the-Future} in support of investment-based discount rates is the Department of Finance’s Handbook of Cost-Benefit Analysis%
    \footcite{Dept-Finance-2006-Handbook-of-CBA}
, which states ``the general conclusion (and a common international practice) is that a producer rate of discount is the appropriate rate of discount to employ.''  Similar to this, \textcite{Harrison-Valuing-the-Future} also cites a 2003 textbook that states ``the usual and safe presumption is that all the capital employed represents investment foregone.'' 

The type of primary empirical evidence required to ascertain the extent of displaced private sector investment is hard to come by. However, it is clear that the case that private sector investment is displaced is far from robust. 

In addition, there remains plenty of debate and disagreement internationally over whether public investment can be thought to displace private investment. \Vref{box:journal-of-CBA-debate-about-discounting} below sets out a debate over these issues published in the Journal of Cost-Benefit Analysis in 2013 that illustrates the contested and ongoing nature of the debate.


\begin{bigbox}{Journal of Cost-Benefit Analysis discount rate debate about the displacement impacts of public investments}{box:journal-of-CBA-debate-about-discounting}
In 2013, the Journal of Cost-Benefit Analysis published a series of papers covering the debate between advocates of differing discounting approaches. Just as was suggested at the start of this chapter, the debate hinged largely on one issue – whether public investments displaced investment or consumption. 

The side of the debate that favoured lower discount rates (based on the social time preference approach), argued that:

\begin{quote}
\textquotedblleft A more realistic assumption is that increased government spending is funded by increased taxes. While some projects at the margin might be financed by debt, all projects are ultimately funded by taxes. Furthermore, taxes primarily reduce current consumption rather than private investment for the simple reason that consumption is much larger than investment (typically five times as large) and, therefore, taxes on investment cannot yield as much as taxes on consumption.”
\end{quote}
The other side argued that higher (opportunity cost of capital) discount rates were appropriate, on grounds that: 

\begin{quote}
\textquotedblleft The appropriate measure … is the social opportunity cost of borrowed funds, not the social opportunity cost of funds raised by an increase in the income tax or some other broad based tax. Admittedly, it is empirically challenging to arrive at reliable estimates of the weights … (i.e., the proportions of an increment of borrowed funds that displace investment versus consumption and, in an open economy, net exports), but the consensus of those who have looked carefully at the matter is that investment is much more sensitive to the rate of return than consumption or net exports, so the bulk of an increment in borrowed funds displaces investment.”
\end{quote}
The challenge in arriving at an empirically-justified solution was clearly well-acknowledged by both sides.

\begin{quote}
\textquotedblleft Because both … approaches are normative and do not generate testable hypotheses that would allow for empirical falsification, the choice of the correct method for choosing a SDR largely depends on what opportunities one believes are sacrificed when a public investment project goes forward: lower taxes and more current private consumption \hl{(the STP-SPC view)}, or lower government debt, lower interest rates and more current private investment (the SOC view).” 
\end{quote}

This debate serves as an illustration of the entrenched views among experts, which have failed to be resolved through empirical analysis.

\boxsources{\textcite{Moore-Boardman-Vining-2013a-More-appropriate-discounting}, \textcite{Burgess-Zerbe-2013-The-most-appropriate-discount-rate} and \textcite{Moore-Boardman-Vining-2013b-The-choice-of-discount-rate}}
\end{bigbox}


\section{A 7 per cent rate isn’t appropriate for all transport projects}\label{section:examples-of-displacement-effects-of-transport-projects}
Transport projects differ in many ways. 

It is widely acknowledged that a single discount rate for all projects does not exist, because the different projects can displace different types of private sector activity and can have different risk profiles.%
    \footcite[][61]{Harrison-Valuing-the-Future}

This section looks at three different types of transport projects in order to assess whether it is at all possible to discern predictable differences in the the displacement effects of each. 

\subsubsection{Western Sydney Airport}
In the 2017-18 Budget, the Australian government announced that it would construct the Western Sydney Airport (WSA) at Badgerys Creek. Unlike many other transport projects, WSA is an example of a government investment in a revenue generating asset.%
    \footnote{Urban Infrastructure Minister Paul Fletcher's comments on the commercially viable nature of the government's WSA and Inland Rail equity investments: Well, if you look at where we are using an equity approach—for example, Western Sydney Airport and airports as we know from the privately owned and operated Sydney Airport, Brisbane Airport, in fact, all of Australia's major airports—these are very viable private businesses. If you look at, for example, the Moorebank Intermodal Terminal, where the Commonwealth is putting in equity. Again, these are businesses with a significant revenue stream, proven business models. Australian Rail Track Corporation charges fees to train operators to use the rail. Of course, there's an \$8.4 million injection of equity into ARTC for Inland Rail in this Budget (\url{http://minister.infrastructure.gov.au/pf/interviews/2017/pfi015_2017.aspx})}
The budget announcement was made following the decision by Sydney Airport Corporation, a private company, not to take up a contractual offer to build and own WSA itself. 

The question relevant to discounting the future flows of economic impacts from the WSA project is to ask what would have happened if WSA did not proceed: would the resources needed to build the WSA have otherwise been used by the private sector to invest more, or to consumed less? 

There are several ways to think about this. 

On the one hand, it is likely that the construction WSA will put pressure on wages and the physical inputs into its construction. For example, to entice workers and capital to work on the project, it may be necessary to pay enough to attract workers and physical inputs away from other activities. This suggests the prospect of ``crowding out'', as some investment opportunities are not undertaken because WSA has made the costs of these opportunities too large to be viable.  

However, complicating this picture, this effect also depends on the extent to which the resources that will be used in the construction and operation of WSA are currently being used in equally productive activities. 

Ultimately it is very hard to know.  

The counter-factual in this case was – very nearly, if SAC took up its contractual right – a private airport expecting to earn a return on its invested capital. And, from a commercial investment we would expect a commercial rate of return. So while this example seems to show that public investment is crowding out private investment, it is not possible to say this with a reasonable degree of certainty.


[\hl{We should also address here the market risk profile of WSA -- would we expect its benefits to be fluctuate evenly with the market, or to be more or less variable?}]

[\hl{Is the best we can do to simply assume that displacement occurs in the ratio of I and C in national income?}]

\subsubsection{Urban road and rail infrastructure projects}
\hl{The key message here I think is that it is generally not possible to make good guesses about the displacement effects of most transport projects.}

Another point we might want to make is that cost-benefit analysis is a partial equilibrium framework -- i.e. it mostly ignores second round effects and relies a lot on the assumption of ``all else equal''. This works best for smaller projects. For large and mega projects, their sheer size means that developing a plausible counterfactual -- of what would have happened if the large amount of resources (up to \$20 billion in the case of Westconnex) was not consumed by the project. 

\subsubsection{Truck curfews}

Truck curfews provide an example of government regulations in the transport sector with impacts that extend to people over coming years, meaning that to evaluate the merits of truck curfews we require a discount rate. 

On first consideration, there are compelling reasons that such regulatory projects likely have little impact on private sector investment: if the regulation shifts trucks from one street to another, the impacts are likely to be very small increases in costs for truck companies who presumably must travel a slower route. But even in this case, if these tiny cost increases are passed on to consumers in the form of higher prices for transported goods, then consumers' lower disposable incomes will result in reductions in consumption or investment, or a combination of both.% 
    \footnote{We also note that truck curfews could require trucks to travel on tolled roads rather than untolled suburban streets, in which case the costs to the trucking companies, and the final impact on consumers, will be larger.}
In this case, it seems reasonable to assume that the proportions of investment and consumption foregone are similar to the proportion of investment and consumption in national income, i.e. around [25] per cent and [75] per cent, respectively [\hl{We should test this proposition with Jim}].%
    \footnote{ABS National Accounts}

\hl{How do the estimated benefits of truck curfews -- such as quieter streets and fewer accidents -- vary with the overall market returns, or with aggregate consumption? Do we acknowledge that slower economic growth will be fewer trucks and so lower noise and accident impacts?  If we don't think they vary greatly, then it would be incorrect to assume that the average market risk premium is appropriate to use.} 

Lessons from this are that, in many cases, regulation is its own category, and governments should on some occasions impose different requirements on different projects. Truck curfews involve relatively little capital costs from either the public or private sectors, so an investment rate is inappropriate here. Further, most of the benefits are intangible, for example in the form of less noisy communities and fewer accidents, which cannot be reinvested. That the investment rate then becomes inappropriate might be dismissed as both obvious and beyond scope – if the scope if taken to be the appropriate discount rate for major capital projects in the transport sector. But it is important, because discounting extends beyond capital projects into all other areas of evaluation that have costs and benefits spread over time, and because in many cases we should be seeking to simultaneously evaluate and compare different options to achieve the same goal. These different options may entail different capital displacement effects. 


\section{Reasons that the discount rate might be closer to the consumption rate of discount}\label{section:downside-pressures-on-a-weighted-discount-rate}

The discussion in \Cref{section:Public-investment-may-not-displace-private-investment} and \Cref{section:examples-of-displacement-effects-of-transport-projects} above has highlighted the problematic nature of assuming that transport projects predominantly displace private sector investment. 

The arguments here seem relatively clear: the evidence base for existing practice is non-existent, and the ability to discern the displacement effects of different projects is very difficult. 

In this section, we note a range of reasons that the discount rate might be closer to the consumption rate of discount. In advancing these arguments, we emphasise that we are straying outside what we can be said definitively. 

A list of reasons the discount rate might be closer to the consumption rate of discount includes:

\begin{itemize}
    \item The following point is made by both \textcite[29]{Grimes-Beyond-Simple-CBA} and \textcite[142]{Arrow-et-al-1995-IPCC-note-on-discounting}: it is frequently argued that if the private sector could achieve the optimum return on its investments then this is the appropriate required rate of return since an alternative for the current generation is simply not to proceed with a certain infrastructure investment (that may yield only 7\% return) when leaving the funds with the private sector instead returns 8\%. Nevertheless, in this situation there is still no guarantee that the grandchildren’s generation will benefit from the private sector’s expenditure choices. Its investments may provide a larger return for the children but that may not be available for subsequent generations. For instance, a future government may tax the children’s returns to provide short-term election-driven consumption benefits. If the current generation cares about future generations and has concerns that the private or public sectors over coming decades may be profligate (relative to the current generation’s concerns for the future) then locking in long-lived infrastructure investments is one method it can use to protect the welfare of future generations. 
    
    \item \textcite[142]{Arrow-et-al-1995-IPCC-note-on-discounting} states that we would expect the marginal return to capital to be less than the average return. The returns to K that we present in Figure X.X in Chapter 2 shows the average returns, not the marginal returns, which we would expect to be lower. 
    
    \item \textcite{treasury2010link-between-fiscal-policy-and-interest-rates}: ``The empirical literature focussing on the link between fiscal policy and interest rates in Australia is relatively scant.''
    
    \item MBV (2) page 402: We think it unlikely that there is much crowding out of anything during periods of less-than-full employment. We also ignore the possibility that the public investment will simply crowd out net exports (as in Lind, 1990).
    
    \item \textcite[404]{Moore-Boardman-Vining-2013b-The-choice-of-discount-rate}: ``If public investments are tax-financed on the margin, then we concur with Arrow (1995) and others that income taxes will primarily affect consumption (since most income is spent on consumption) and that other taxes will fall even more heavily on consumption.''

\end{itemize}




\chapter{What should Australia do?}\label{chap:what-should-Australia-do}

The challenge for policymakers is not simply a task of translating accepted academic conclusions into the policy sphere, because even among academics many of the issues involved remain, at best, resolved.%
    \footcite[][24]{Grimes-Beyond-Simple-CBA}
Instead, the challenge for policymakers is about finding a practical and pragmatic path through the complex literature, and being transparent about the limitations of our empirical and theoretical understanding.

Summary of the discussion in chapters 1-3:
- Very hard to find strong evidence
- There are reasons that the discount rate for different projects should be different -- i.e. one-size-fits-all is unrealistic
- Conclude with the comment that what we need is a pragmatic path through these complex issues

We feel confident with our concerns about the lacking evidence base for the displacement of capital. 

But across a range of other relevant issues, where there is insufficient evidence for us to be definitive, we feel there is further downward pressure on the choice of discount rate. For example:
\begin{itemize}
    \item The treatment of risk, in particular whether government should act risk neutrally rather than being risk averse, as most individuals are. 
    \item We have also not seen a body of evidence to suggest that, as Harrison (2010, page 60) claims, `'In the absence of information on the quantity of risk in a government project, it is reasonable to assume that the average government project is no less risky than the average private investment. The consensus view is that most government projects are highly correlated with returns to the economy as a whole.'' 
    \item Government investment could in some cases create, not displace, private investment. 
    \item Grimes material at the top of page 29 about how ``if current generations care about future generations and has concerns that the private or public sectors over the coming decades may be profligate (relative to the current generation's concerns for the future) then locking in long-lived infrastructure investments is one method it can use to protect the welfare of future generations.''
\end{itemize}

Academia also does not offer us a path through, as Box XX sets out the long-held disagreements between academic experts. 




\section{Discount rates should be lower}
While true that there is no single correct discount rate that covers all project types,%
    \footcite[][1]{Grimes-Beyond-Simple-CBA}
defining one central discount rate remains an important undertaking. Allowing project proponents to choose their own rates raises significant risks for incorrect usage that would be difficult to police. Deviations from the one central discount rate should occur, but deviations should be exceptions rather than normal practice. Deviations should also be done in consultation with key stakeholders, including the public. The best way to conduct this consultation with the public is to make all business cases and investment analysis entirely transparent. 



\subsubsection{Recommendation -- lower discount rates}

\subsubsection{Recommendation -- investigate impact of lower discount rates}


\section{Tighten project appraisals}

\subsubsection{Recommendation -- publish business cases}
transparency matters a lot, especially given that the appropriate rate should vary somewhat across different projects – without transparency, poor practice goes unnoticed.
Indeed, it is equally concerning that excessively low rates be used where they cannot be justified, making government value even commercial projects more highly than the private sector does. 

\subsubsection{Recommendation -- tighten project appraisal}
We know from our own research into cost overruns in transport projects that risk is not well-handled in project assessments.

This is an important component of getting around the concern that lower default discount rates provide the government with incentives to own far more capital, including commercial interests, than is likely to be optimal.%
    \footnote{Quiggin, 2005, Risk in CBA (BTRE): ``Evaluation of public-sector projects at a risk-free discount rate significantly lower than rates used by the private sector for financial analysis could raise concerns about government investment crowding out private-sector investment. However, addressing downside risk for public-sector projects should work in the opposite direction. Also, the private sector has other offsetting advantages; and overall levels of government investment are budget-constrained.''}

