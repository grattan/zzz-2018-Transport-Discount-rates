\chapter{What is the discount rate}\label{chap:what-is-the-discount-rate}

The discount rate is a device for putting costs and benefits today on a comparable footing with costs and benefits in the future. It is fundamental to economic evaluation, with applications to all public policies where impacts on future generations need to be valued. Both the United Kingdom’s%
    \footnote{Stern, N. H. 2007. The economics of climate change: the Stern review. Cambridge, UK: Cambridge University Press.}
and Australia’s%
    \footnote{Garnaut, R., 2008. The Garnaut climate change review: final report. Cambridge University Press.}  
landmark assessments of the need for action on climate change hinged on the choice of discount rate.%
    \footnote{Nordhaus, W.D., ‘A Review of the Stern Review on the Economics of Climate Change’, Journal of Economic Literature, Vol. XLV (September 2007), p. 686.} 

Although it has a major impact on economic evaluation, the discount is also a complex and poorly understood variable that policymakers and the public should have a greater understanding of. In part, this relates to the general problem of transparency in economic evaluation. 


\section{This report is designed to help navigate complexity}

Arguments about discounting are hard for many policy makers to navigate, because the field is technical and complex. In addition, the low level of transparency of economic evaluation of transport projects and programs obscures from public view the role that the discount rate plays. 

We have written this paper for policy makers at the Commonwealth and state level who determine the discount rate that is used as part of the economic evaluation of transport infrastructure proposals. The intention is to illuminate and de-mystify the discount rate, and to examine the evidence on which current practice is based.

The challenge for policymakers is not simply a task of translating accepted academic conclusions into the policy sphere. As an academic topic, there remains plenty for academics to argue over. Nonetheless, the the importance of discounting to how and why we govern means that a practical and pragmatic path through the complex literature is necessary.

We believe that the discount rate used for transport infrastructure projects should be lower. This view is based on an examination of theory and empirical evidence. However, as we emphasise in Chapter 4, if a different approach to discounting were adopted, further changes to project appraisal would be needed to mitigate some undesirable effects.

The rest of this section explains in more detail why the discount rate matters and the confused way it is currently set in Australia.


\section{The choice of discount rate matters a lot}

The discount rate matters because the costs and benefits of a project accrue at different points in time. An investment in a new bridge, for example, will have construction costs in the first one or two years and benefits following in the years once construction has been completed. 

A future benefit or cost needs to be revalued to convert it into today’s dollars, because future dollars have a different value to today’s dollars – even ignoring the effects of inflation. It’s common to assume that a dollar today is worth more than a dollar at some future date. For example, a dollar today can be converted into more than a dollar in the future through investment -- it is preferable to have a dollar today, rather a dollar in ten years’ time, because that dollar today can grow into more than a dollar in ten years’ time. 

We can see the diminishing value of a dollar over time in people’s decisions about saving -- by deferring consumption today it is possible to have higher consumption in the future. 

On the basis that future dollars are worth less than current dollars, a higher discount rate makes more distant impacts smaller than a lower discount rate does (see Box 1 for an explanation of the practical mechanics). Because infrastructure projects typically have costs occurring today and benefits arising in the years thereafter, a lower discount rate imposes a smaller penalty on the benefits than it does on the costs. As such, using a lower discount rate makes the economic merits of a project appear stronger than they would if a high discount rate was used. 

The reason this matters is not because it affects the size of the infrastructure portfolio or government investment decisions from year to year. Generally, the size of the investment in any given budget is largely established top-down as part of the overall budget strategy rather than bottom-up from the economic appraisal of individual infrastructure projects.

But it does matter for project selection, in two ways.

The first way is that it affects the priority ordering of project proposals. Figure 1 illustrates how discount rates of 4, 7 and 10 per cent have quite different impacts on several major current projects. The Melbourne Metro and Inland Rail project are much more sensitive to the choice of discount rate than Canberra’s Capital Metro or Victoria’s Murray Basin Rail projects. This is because much of the benefit of Melbourne Metro and Inland Rail is expected to occur well into the future.

\begin{smallbox}{The mechanics of discounting}{box:<cross-ref key>}
A typical economic evaluation of a transport project looks at the annual impacts of an investment decision over a 30-100 year period. The chosen discount rate diminishes future values by specifying exactly what discount factor needs to apply to the costs and benefits that arise in each year. 

Discount factor in $"Year  n"=  1/[(1+Discount rate)]^n$ 

The impact of the discount rate on \$100 arising in a range of future years is:%
    \footnote{Note that these adjustments are made after adjusting for inflation. 

    While the typical transport project evaluation looks at a 30-50 year period, the inclusion of ‘residual value’ benefits – i.e. the calculation of any remaining asset value at the end of the evaluation period – means that the discount rate used in transport projects often applies for up to 100 years.}

\begin{itemize}
\item Using a 0 per cent discount rate,\$100 in 50 years’ time is valued at \$100 when we value it today.
\item Using a 3.5 per cent discount rate,\$100 in 50 years’ time is valued at \$18 when we value it today.
\item Using a 7 per cent discount rate,\$100 in 50 years’ time is valued at \$3.39 when we value it today.
\item Using a 10 per cent discount rate, \$100 in 50 years’ time is valued at 85 cents when we value it today.
\end{itemize}

\end{smallbox}

When a project’s benefits are further into the future, a high discount rate treats those benefits as smaller than a low discount rate would. A higher discount rate therefore directs decision makers towards projects with nearer term benefits. 

The second reason the discount rate matters for project selection is its impact on the assessment of a project’s viability. A change in discount rate can determine whether, using the economic evaluation as our guide, the project is worth building or not worth building. While each of the four projects in Figure 1 has benefits greater than costs at a discount rate of 4 per cent, a discount rate of 7 per cent makes Inland Rail and Melbourne Metro rather marginal, and at a discount rate of 10 per cent, building them would destroy value. 

The projects illustrated in Figure 1 are those with sufficiently detailed business cases in the public domain to provide visibility of sensitivity testing. Business cases are not available publicly for the overwhelming majority of transport infrastructure projects, which severely limits our capacity to apply an understanding of discount rates so as to explain the real-world impact of different discount rates.  

\begin{figure}
\caption{Benefit-cost ratios of major infrastructure projects with respect to discount rate \label{<cross-reference key>}}%
\units{Benefit-cost ratio}
\includegraphics[page=4]{Charts/ChartPackDiscountRates.pdf}
\noteswithsource{Capital Metro results include “wider economic benefits.”}%
{Capital Metro Business Case (2014, p. 16), Murray Basin Rail Project Business Case (2015, p. 128), Melbourne Metro Business Case (2016, p. 185), Inland Rail Business Case (2014, p. 186).}
\end{figure}

When a project’s benefits are further into the future, a high discount rate treats those benefits as smaller than a low discount rate would. A higher discount rate therefore directs decision makers towards projects with nearer term benefits. 

The second reason the discount rate matters for project selection is its impact on a project’s viability. A change in discount rate can make the difference between a project being worth building and not worth building. While each of the four projects in Figure 1 has benefits greater than costs at a discount rate of 4 per cent, a discount rate of 7 per cent makes Inland Rail and Melbourne Metro rather marginal, and at a discount rate of 10 per cent, building them would destroy value. 

The projects illustrated in Figure 1 are those with sufficiently detailed business cases in the public domain to provide visibility of sensitivity testing. Business cases are not available publicly for the overwhelming majority of transport infrastructure projects, which severely limits our capacity to apply an understanding of discount rates so as to explain the real-world impact of different discount rates.  

\section{Australian discounting practice is confused}
Perhaps rightly, given the complexity of the topic, discounting practice in Australia is not entirely coherent. To take a clear example of this, many commentators assume that because government borrowing rates have fallen over the last decade that the opportunity cost of government investments have therefore fallen.%
    \footnote{Cite examples of this view in the media - Colebatch and Davidson are possible places to look for it.}
But this is symptomatic of the broader confusion over discounting, as \Cref{box:borrowing-rates-are-not-the-discount-rate} explains. 

\begin{smallbox}{Why the government borrowing rate is not the discount rate}{box:borrowing-rates-are-not-the-discount-rate}

[If we need to use concepts like investment and consumption rates of interest, then this belongs in chapter 2. But if it only refers to opportunity cost, then it can fit here]

[Should be able to make use of Cowen's 2016 blog post here]

[Harrison also covers this point] 

[Boardman also covers this point] 

\end{smallbox}

Most Australian states and the Commonwealth require a discount rate of 7 per cent for appraising transport infrastructure projects. This consensus is a recent phenomenon: as lately as 2013 there were highly disparate discounting practices across jurisdictions.%
    \footnote{Argyrous, G., A review of government cost-benefit analysis guidelines, SSC/ANZSOG Occasional Paper, Australia and New Zealand School of Government, March 2013.}

Although most jurisdictions use 7 per cent, their stated reasons for doing so vary. The reasons for adopting 7 per cent appear to be a combination of a pragmatic reading of the complicated literature, and the influence of a key report by Mark Harrison published in 2010.%
    \footnote{Harrison, M. 2010, Valuing the Future: the social discount rate in cost-benefit analysis, Visiting Researcher Paper, Productivity Commission, Canberra}

Infrastructure Australia justifies its choice of discount rate (for projects seeking a Commonwealth funding contribution) by referring to the advice of state governments and several other relevant discounting authorities.%
    \footnote{Infrastructure Australia, Assessment Framework Detailed Technical Guidance, infrastructureaustralia.gov.au, Sydney, January 2016, page 37.}
There is a degree of circularity in this, however, because most states themselves defer to Infrastructure Australia in selecting a discount rate – most obviously in the economic evaluations of projects that are submitted to Infrastructure Australia for assessment (Figure 2). 

This may be a reasonable response to the challenging technical literature on the topic (discussed in the following chapter). However, it also suggests an absence of leadership on the topic – given that the argument supporting the use of different rates in different jurisdictions within Australia is thin.

Of course, this only matters to the extent that a positive economic evaluation is a requirement for a project to be chosen and funded by government. Unfortunately, this is not always what happens.%
    \footnote{Terrill, M., Emslie, O. and Coates, B. 2016, Roads to riches: better transport investment, Grattan Institute.}
But if project appraisal practices do meaningfully determine project selection -- which by-and-large they should -- the choice of discount rate can be a make-or-break decision. 

\begin{figure}
\caption{Formal sources of discounting guidance \label{<cross-reference key>}}%
\units{}
\includegraphics[page=5]{Charts/ChartPackDiscountRates.pdf}
\noteswithsource{the Transport and Infrastructure Council’s guidance is provided through its responsibility for the Australian Transport Assessment and Planning (ATAP) guidelines}%
{Grattan analysis of State and Commonwealth government reference documents on cost-benefit analysis – details in this working paper’s reference list.}
\end{figure}
