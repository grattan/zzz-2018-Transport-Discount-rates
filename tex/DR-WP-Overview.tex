\begin{overview}

The 2017-18 Budget had the nation’s economic and transport policy wonks talking. How precarious are economics of the Government’s major investment in the Inland Rail freight project?

Answers bring to life one of the least understood but most critical economic parameters, the discount rate. The discount rate is literally the "discount" that is applied to future values -- in order to compare costs and benefits today with costs and benefits in the future. It’s important because it describes how we share resources between our own generation and generations in the future. Such a concept is ubiquitous in the formulation of public policy. 

The choice of discount rate is critically important for the merits of Inland Rail. The Australian Rail Track Corporation, using a 4 per cent discount rate, says the project will deliver benefits 2.6 times greater than the \$10 billion cost. But using the 7 per cent discount rate recommended by Infrastructure Australia, the estimated benefits may be as low as 2 per cent larger than the costs – if all goes to plan. 

On one reading, then, it is expected to be a major economic success, but on another, its value is highly marginal. 

Discounting is complex subject matter -- it remains one of the great unresolved issues in economics. This working paper does not provide a new academic contribution. Its focus is to illuminate concepts and examine the evidence on which the application of discounting is based. 

Despite the numerous variants in language and nuance, approaches to discounting can be divided into two schools of thought. The difference largely hinges on the question of what would have happened instead of the public infrastructure project in question: in what proportion would people and firms have invested less or consumed less as a consequence of public sector investments?

If private sector investments are forgone when we undertake transport projects, then we should discount the impacts of the project using something close to an average private sector capital return. 

But if private investment is not diminished by a public project -- if the impact is instead largely that people forgo consumption, then we should use a lower rate, such as the after-tax return on savings. 

The primary evidence needed to resolve these issues is hard to come by. But existing practice for transport infrastructure, which has converged around Infrastructure Australia’s requirement to use 7 per cent, is inappropriate -- because our close look at the evidence behind existing practice finds it to be threadbare, at best.

We recommend taking a pragmatic path through the issues. This can be done by balancing the available evidence. Taking a mid-point of the "investment" and "consumption" approaches suggests using a rate of around 5 per cent. Sensitivity testing above and below this is important.  

Moving to a lower discount rate will make the economics of all transport projects look better. We are somewhat uncomfortable about this. Grattan analysis has previously shown that, in recent history, Australia has systematically underestimated the cost of projects, making many appear more attractive than they really were. In a world where politicians are too often tempted to waste public money on projects that are not very important to the economy but popular with local interests, the use of a higher discount rate has served as a useful counter-balance. 

But the bar should be higher: we should seek to improve all aspects of project evaluation, and to make transparent those areas where ambiguity remains. Without this, the wrong projects become priorities and receive too little scrutiny. 

\end{overview}