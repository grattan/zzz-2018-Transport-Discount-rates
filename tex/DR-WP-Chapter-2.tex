\chapter{Two approaches to discounting}\label{chap:two-approaches to discounting}

\section{Valuing benefits and costs that arise in the future}

[\hl{Tim: Something is missing in the opening here. How does what is foregone relate to how we should convert future benefits into current dollars?}]

When we employ a discount rate, we are trying to get a sense of how people -- and how societies -- place a monetary value on things that happen in the future. As such, the challenge of discounting is to find a reasonable \emph{rule} that allows us to adjust impacts that occur in the future. 

There are two dominant approaches to this seemingly fraught exercise.   

As Chapter 1 discussed, society must ``give up" something when public projects are undertaken. 

This provides an insight into how future impacts might be valued -- i.e. what the appropriate rate of discount should be.  

Despite the numerous variants in language and nuance, approaches to discounting can be divided into two schools of thought. The difference largely hinges on the question of what would have happened instead of the public infrastructure project in question: in what proportion would people and firms have invested less or consumed less as a consequence of public sector investments? 

Sections 2.1 and 2.2 below describe the two main analytical discounting approaches. Both of these have their own lengthy list of variations and sub-types. The purpose here is to illuminate some of the fundamental issues and points of difference, rather than to describe the literature exhaustively. 

\section{Cost of capital approaches}
The ‘opportunity cost of capital’ approach is based on the idea that public infrastructure projects typically leads to the private sector investing less than it otherwise would. This perspective therefore infers the discount rate from private sector returns to capital. It has led to discount rates in the range of 6-8 per cent.%
    \footnote{Argyrous, G., 2013}
The current use of 7 per cent as the central discount rate relies on this concept. 

The reason for the dominance of this approach in Australia appears to have been the influential 2010 report by Mark Harrison, written while with the Productivity Commission as a visiting researcher. Harrison’s clear and thorough explanation of the cost of capital approach makes a case that “market rates reflect the opportunity cost of investing in public projects, and there is no case for allocating resources to low return investments when higher returns are available.”%
    \footnote{}
However, because there is a difference between the before-tax ‘investment rate of return’ that investments earn and the after-tax ‘consumption rate of return’ that lenders receive, the discount rate should reflect the extent to which financing is expected to reduce investment and consumption. In this sense, Harrison recommends a ‘weighted’ cost of capital approach, rather than assuming that public sector projects entirely displace private sector investment. 

Despite employing a ‘weighted’ cost of capital approach, however, Harrison’s assessment is that public sector projects almost entirely displace private sector investment, and the appropriate discount rate should therefore lie close to the before-tax return on private sector capital.%
    \footnote{}

Harrison's approach also assumes that the riskiness of the average public sector project is comparable to that of the average private sector project.%
    \footnote{“Riskiness” here refers to a project’s systemic, or non-diversifiable, risk. This is the risk associated with a project’s returns fluctuating with the macroeconomic cycle. It does not refer to project-specific risk, which is handled separately through the use of ‘certainty equivalent’ estimates of costs and benefits.}
Because the relationship between the discount rate and riskiness of a project is an important and often misunderstood aspect of the literature, Box 2 sets out the concepts and issues in more detail. 

\begin{smallbox}{The relationship between the discount rate and risk}{box:<cross-ref key>}
Sources - 
Arrow and Lind
Quiggin on costings (for BITRE)
Mark Harrison
Paper that sets out that most risk is market risk

Question here is whether to include existential risk in this box -- or whether to add it to the footnote on the Ramsey equation.  

\end{smallbox}

\section{Social time preference approaches}

\hl{Ramsey equation section below can draw straight from Summers and Zeckhauser}

‘Social time preference’ based discount rates adopt the premise that public sector projects reduce consumption rather than investment. The main analytical approach uses the ‘Ramsey equation’ to determine the discount rate, based on concerns with the use of markets to infer discount rates.%
    \footnote{The reasons for concern with market-inferred discount rates are well set out in Parker, C.,The implications of discount rate reductions on transport investments and sustainable transport futures, NZ Transport Agency research report 392, December 2009, page 30.}

The Ramsey equation is formally the basis for discount rates used in the United Kingdom. It constructs the discount rate by looking at the ‘pure time preference’ rate – individuals’ preference for consumption now rather than later, with an unchanging level of consumption per capita over time. In other words, how much do we prefer things today rather than tomorrow simply because [...]

The Ramsey equation also factors in the expectation that future generations will have greater levels of consumption, and that the additional increment of ‘utility’ they receive from consumption diminishes as consumption grows over time.%
    \footnote{The Ramsey equation is typically presented as: \(\text{Discount Rate} = \rho + \mu\,g\), where \(\rho\) is the is the rate at which individuals discount future consumption over present consumption, on the assumption that no change in per capita consumption is expected (``pure'' time preference), \(\mu\) is the elasticity of marginal utility of consumption with respect to utility, and \(g\) is the expected annual growth rate of consumption per capita.}

An alternative social time preference approach uses market returns to determine the appropriate rate of discount. It looks at the best return that many individuals are able to earn in exchange for postponing consumption (the “consumption” rate of discount), which in many cases is the real after-tax return on savings.%
    \footnote{Boardman, AE., Greenberg, DH., Vining, AR., Weimer., DL. Cost-Benefit Analysis: concepts and practice (2nd Edition), Prentice Hall, 2001, page 239.}

In practice, the social time preference approach leads to discount rates in the range 2-4 per cent.  

Importantly, under the social time preference approach, where it is a private sector investment that will be displaced by a public sector project, this needs to be incorporated into the analysis. This is done on a project-by-project basis by applying an additional weighting, known the ‘shadow price of capital’, to any estimates of displaced private sector investment.%
    \footnote{Boardman et al,  page 239.}

One situation where a social time preference approach would seem appropriate where some new transport infrastructure has intangible benefits; that is, where there are benefits that people use and enjoy but cannot store up or reinvest for the future.  An example of an intangible benefit would be a quicker and more comfortable holiday trip on a new road; by contrast, a more efficient work-related trip has not only the same benefits as the holiday trip, but also benefits arising via the (small) increase to that worker’s economic contribution, which also benefits future generations. This example also helps to illustrate the complexities involved in assessing the extent to which consumption or investment is forgone when public projects are undertaken. 

\begin{smallbox}{The relationship between the discount rate and risk}{box:discount-rate-and-risk}
Sources - 
Arrow and Lind
Quiggin on costings (for BITRE)
Mark Harrison
Paper that sets out that most risk is market risk

Question here is whether to include existential risk in this box -- or whether to add it to the footnote on the Ramsey equation.  

\end{smallbox}

[Revisit Grimes -- I think what he is really saying is not that the benefits are intangible so we should use a consumption rate of discount, but more that when the benefits are intangible we shouldn't assume they can be reinvested, and indeed that our ability to compare these future impacts in today's terms becomes very difficult. This ties in with the Cowen view in Stubborn Attachments that "if you are doing valuation in terms of dollars, and the funds will in fact be reinvested, the standard approach outlined in the blog post is correct. If you are considering well-being, in the context of macro social choice, then I think Stubborn Attachments is correct."]

Much public sector investment in transport infrastructure clearly has intangible as well as tangible benefits, and so a social time preference approach offers a way to consider the rate at which society is willing to trade those benefits today with benefits in future.
